\chapter{Introduction}\label{sec:introduction}

The discovery of particles at the electroweak scale, such as the top quark at Fermilab's CDF and D0~\cite{topD0,topCDF} and the Higgs boson at the Large Hadron Collider (LHC) in CERN~\cite{higgscms,higgsatlas}, led to discovery of all constituents in the Standard Model (SM). 
The SM describes the nature of fundamental particles and their interactions with precision. 
In spite of its success, the SM suffers from the few obstacles:
the evidence of neutrino masses and mixing~\cite{neutrino}, the obsevations of bullet clusters confirming the presence of dark matter (DM)~\cite{Baumgart:2009tn,Kaplan:2009ag,Chan:2011aa,Dienes:2011ja,Dienes:2012yz}, and baryon-antibaryon asymmetry~\cite{Cui:2014twa} all remain unexplained in the framework of SM. 
In addition, the SM suffers from the naturalness problem. 
To solve such issues, one needs to look for physics Beyond the Standard Model (BSM).

The naturalness problem originates from the fact that the SM Higgs is a scalar particle. 
Unlike fermions or gauge bosons, its mass is not protected by any symmetry and is subject to large radiative corrections, especially from the top quark loop. 
Thus, for the SM to be valid up to the Planck or Grand Unification Theory (GUT) scales, the necessary radiative corrections are enormous. 
One needs an exorbitant amount of fine-tuning to fit the Higgs mass at the observed value of 125GeV.
One of the most popular solutions to this problem is Supersymmetry (SUSY), which assigns chirality to the Higgs particle. 
SUSY solves the fine-tuning problem, neutrino masses, and provides a candidate for DM. 
Unfortunately, the LHC has found no significant excess over the SM background in their search for SUSY\cite{SUSY}. 
Although the non-observation of supersymmetric partner particles does not invalidate SUSY, it makes less attractive among the particle physics community. 
Non-observation of superpartners, particularly the stop (scalar partner of the top quark) has pushed its mass beyond 1TeV. 
This generates "little hierarchy" problem, but an alternative solution of "neutral naturalness" remains. 

In the framework of neutral naturalness, the top partners are not charged under the SM color group. 
Because of being colorless, their production crosssection is much smaller, and the present limits on the top partner particles are well below 1TeV. 
Examples of neutral naturalness models are the Twin Higgs \cite{Chacko:2005pe},
Folded SUSY \cite{Burdman:2006tz}, and the Quirky Little Higgs \cite{Cai:2008au} models.
Theoretical models provide the possibility of neutral Long-Lived Particles (LLPs), which may be produced in the proton-proton
collisions of the LHC, and decay back to SM particles far from the interaction point (IP).\cite{Craig:2015pha}
If the mirror QCD gluons form scalar glueballs, the SM Higgs boson can become a "Higgs Portal" between the SM and BSM mirror QCD scalar glueballs. 
In the Mirror SM and Twin SM models, only the SM Higgs boson can interact with both SM QCD and mirror QCD particles.
BSM mirror QCD scalar glueballs can only decay back to SM particles via Higgs boson decay as well. 
Because of its decay as an offshell Higgs boson, its crosssection is highly suppressed. 
Decay branching ratio to highest mass fermions will be highest following the Yukawa couplings.
Decay ratio into b quarks or tau leptons are highest depending on the mirror scalar's mass.
The displaced decays of the scalars would lead to exotic signatures in the CMS, such as distant innermost tracker hit, displaced vertices, and displaced jets.
Phenomenology of long-lived particles in LHC entailed increase in interest of neutral naturalness framework among the particle physics community. \cite{Curtin:2015fna,Csaki:2015fba}.
The long-lived scalar model is shown in the Figure (to be made soon).

Searches for LLPs decaying into final states containing jets were investigated
at the Tevatron ( $\sqrt{s}$ = 1.96~TeV) by both CDF~\cite{Aaltonen:2011rja} and D0~\cite{Abazov:2009ik} Collaborations,
at the LHC by the ATLAS and LHCb Collaborations at $\sqrt{s}$ = 7~TeV~\cite{ATLAS:2012av,Aaij:2014nma},
by the ATLAS, CMS and LHCb Collaborations at $\sqrt{s}$ = 8~TeV~\cite{Aad:2015uaa,Aad:2015rba,PhysRevD.91.012007,Aad:2015asa,Aaij:2017mic,Aaij:2016xmb,Aaij:2015ica}.
More recently, by the CMS ~\cite{Sirunyan:2017jdo,displacedvertices,displacedjets2016,delayedjets,emergingjets,CMS-PAS-EXO-19-021}
 and ATLAS Collaborations ~\cite{Aaboud:2018iil,Aaboud:2018jbr,Aaboud:2018arf,Aaboud:2018aqj,Aaboud:2018kbe,Aaboud:2019trc,Aaboud:2019opc,Aad:2019kiz,Aad:2019pfm,Aad:2019tcc,Aad:2019xav,Aad:2019tua} at $\sqrt{s}$ = 13~TeV. 
CMS Collaboration released a new result in 2021, in which the Higgs is created in association with a Z vector boson ~\cite{ZHAN}, for a better probe into lighter scalar masses thanks to its clean dilepton trigger.

Although exclusion limit on branching ratio of the Higgs to the LLPs to b and d-quark were set below unitarity for analyses above, exclusion limit for $\tau$ final state has been frequently omitted or presented with values above 1.  
Displaced Jets analyses face challenges for $\tau$ final state, due to $\tau$'s non-trivial hadronic and leptonic decay modes and complicated reconstruction mechanism. 
However, the Leptophilic model for Twin Higgs and other Higgs models are also highly motivated ~\cite{Lepto}. Continuous neglect of $\tau$ final state limit is not only a poor practice, but overlooks an important unexplored phase space. 
This analysis searches for Higgs Portal model with the Higgs' Leptophilic nature, which focuses on $\tau$ final state.
The 55 GeV maximum scalar mass is set to investigate only on-shell neutral scalar particles from the Higgs. 
A minimum 7 GeV mass for scalar particles is required to create on-shell tau-lepton pairs.
Feynman diagram of the scalar particle production mechanism is depicted in Figure (to be made soon).

Most CMS searches are not optimal for detecting Higgs boson decays to displaced jets
due to the soft $p_T$ nature of its decays products.
%Standard CMS searches rely on $H_T$ triggers that are highly inefficienty for this signal.
Low hadronic activity in transverse signature becomes particularly more difficult with a long-lived signature.
Higgs produced in association with Z vector boson analysis~\cite{ZHAN} overcame this barrier with help of dilepton trigger. 
Although ggH production mode gives the largest Higgs crosssection, it complicates the trigger strategy even further. 
This analysis exploits the $\tau$ lepton's leptonic decay, in which the $\tau$ lepton decays into a soft muon, using a trigger of B Parking High Level Trigger (HLT) Path implemented in CMS for the 2018 portion of Run 2.

Another challenge for a $\tau$ lepton analysis is on the different decay modes of $\tau$ leptons. 
$\tau$ leptons decay hadronically and leptonically, with several different sub-decay modes. 
Developing analysis strategies to optimize the search for each sub-decay modes is extremely complicated, a main reason for the omission or no good exclusion limit in precedent LHC results.
To be inclusive of all $\tau$ leptons' decay modes, a displaced vertex search can be more efficient than a displaced objects (jet, muon, electrons) search. 
We exploit the newly developed Regions of Interest mechanism in the tracker volume. 
Regions of Interest (ROI) form displaced vertex candidates, by fittng pair-wise tracks of Lost-tracks and PackedPFCandidates classes in MINIAOD data into a vertex. 
ROIs save all relevant track and fitted vertex qualities along with isolation information.
These variables are used as input for Machine Learning (ML) algorithms, enabling a highly generic and data-scientific search method.


The rest of the dissertation is organized as follows.
In Section~\ref{sec:theory}, we discuss the theoretical background for the need of the BSM theory in more details. 
In Section~\ref{sec:detectors}, the CMS detector is thoroughly discussed with emphasis on the tracks and the calorimeter, which are relevant detector parts for the analysis. 
We discuss how the analysis exploited the b-parking trigger in Section~\ref{sec:triggers}, with description of its original motive for the trigger's implementation. 
The physics objects and formation of Regions of Interest are described in Section~\ref{sec:objects}.
The machine learning algorithms are futher explained in Section~\ref{sec:machinelearning}. 
The event selections are presented in Section~\ref{sec:selections}. 
Section~\ref{sec:estimate} describes the data driven background estimation method. 
Section~\ref{sec:systs} describes the background estimation method's validation process and systematic uncertainties.
Finally, Section~\ref{sec:results} presents the results of the search.
We conclude with Section~\ref{sec:conclusions}.
