\chapter{Introduction}\label{sec:introduction}

The discovery of particles at the electroweak scale, such as the top quark at Fermilab's CDF and D0~\cite{topD0,topCDF} and the Higgs boson at the Large Hadron Collider (LHC) in CERN~\cite{higgscms,higgsatlas}, led to discovery of all constituents in the Standard Model (SM). 
The SM describes the nature of fundamental particles and their interactions with precision. 
In spite of its success, the SM suffers from the few obstacles:
the evidence of neutrino masses and mixing~\cite{neutrino}, the obsevations of bullet clusters confirming the presence of dark matter (DM)~\cite{Baumgart:2009tn,Kaplan:2009ag,Chan:2011aa,Dienes:2011ja,Dienes:2012yz}, and baryon-antibaryon asymmetry~\cite{Cui:2014twa} all remain unexplained in the framework of SM. 
In addition, the SM suffers from the naturalness problem. 
To solve such issues, one needs to look for physics Beyond the Standard Model (BSM).

To search for the BSM, the high energy physics (HEP) community has completed many researches both in theoretical and experimental sides. 
Theoretical high energy physics community approached the issue with 2 main approaches. 
The first approach tackles the issue of precision. 
Although the SM is very well constrained model, further precision of particles in the SM, especially those in the electroweak scale, gives new insight for the BSM physics.
For instance, the CDF collaboration recently discovered a 7$\sigma$ deviation of the W boson mass from the SM prediction~\cite{Aaltonen:2022aaa}. 
The W boson mass deviation has been interpreted for new physics using the framework of the Standard Model Effective Field Theory (SMEFT)~\cite{Mishima:2022aab}.
In SMEFT, the SM operators from the SM Lagrangian are used to build 5,6,8-dimensional new terms for the Lagrangian. 
The SMEFT scalar, fermion, or vector extension gives hint for new insight for the BSM and its phenomenology~\cite{Mishima:2022aab}.
The other approach aims to build an entirely new Lagrangian term, which can be appended to the current SM Lagrangian term. 
The BSM Lagrangian introduces new particle fields and should address unsolved issues in the current SM framework.
The new particles' mass scale can range from those in the collider level upto the astrophysical level.

To find the new BSM particles, the experimental high energy physics community conducted many different researches based on both collider and astrophysical data.
Experimental physicists invested strenuous effort on data from the collider physics, since lightest particles from popular theory, such as supersymmetry (SUSY), were within the collider's hard scattering energy level.
Experimentalists searching for the BSM particles can also divide their main approaches into 2, the first with prompt decay of the BSM particle and the second with long-lived lifetime signature.
CMS analysis targeting the BSM particles with prompt lifetime have been fully studied and resulted in no deviation from the SM prediction ~\cite{SUSY}.
However, the second approach, which has not been fully investigated, is when particles decay with Long-Lived signature, in other words, Long-Lived particles (LLP).
This signature is uniquely interesting and challenging for scientists.
It requires different analysis strategy depending on the mass scale (MS) and lifetime (c$\tau$) of the BSM particle.
Thus, this frontier has been perceived as the blue ocean for HEP experimentalists, to the extent that a new detector solely targeting LLPs is planned to be built ~\cite{Barron:2022aac}.

In this dissertation, we focus on the LLPs originating from the LHC, specifically the CMS, review precedent analyses, and propose a novel strategy. 
Searches for LLPs decaying into final states containing jets were investigated
at the Tevatron ( $\sqrt{s}$ = 1.96~TeV) by both CDF~\cite{Aaltonen:2011rja} and D0~\cite{Abazov:2009ik} Collaborations,
at the LHC by the ATLAS and LHCb Collaborations at $\sqrt{s}$ = 7~TeV~\cite{ATLAS:2012av,Aaij:2014nma},
by the ATLAS, CMS and LHCb Collaborations at $\sqrt{s}$ = 8~TeV~\cite{Aad:2015uaa,Aad:2015rba,PhysRevD.91.012007,Aad:2015asa,Aaij:2017mic,Aaij:2016xmb,Aaij:2015ica}.
More recently, by the CMS ~\cite{Sirunyan:2017jdo,displacedvertices,displacedjets2016,delayedjets,emergingjets,CMS-PAS-EXO-19-021}
 and ATLAS Collaborations ~\cite{Aaboud:2018iil,Aaboud:2018jbr,Aaboud:2018arf,Aaboud:2018aqj,Aaboud:2018kbe,Aaboud:2019trc,Aaboud:2019opc,Aad:2019kiz,Aad:2019pfm,Aad:2019tcc,Aad:2019xav,Aad:2019tua} at $\sqrt{s}$ = 13~TeV. 

CMS Collaboration released a new result in 2021, in which the Higgs is created in association with a Z vector boson ~\cite{ZHAN}, for a better probe into lighter scalar masses thanks to its clean dilepton trigger.
Although exclusion limit on branching ratio of the Higgs to the LLPs to b and d-quark were set below unitarity for analyses above, exclusion limit for $\tau$ final state has been frequently omitted or presented with values above 1.  
Displaced Jets analyses face challenges for $\tau$ final state, due to $\tau$'s non-trivial hadronic and leptonic decay modes and complicated reconstruction mechanism. 
However, the Leptophilic model for Twin Higgs and other Higgs models are also highly motivated ~\cite{Lepto}. Continuous neglect of $\tau$ final state limit is not only a poor practice, but overlooks an important unexplored phase space. 
This analysis searches for Higgs Portal model with the Higgs' Leptophilic nature, which focuses on $\tau$ final state.
The 55 GeV maximum scalar mass is set to investigate only on-shell neutral scalar particles from the Higgs. 
A minimum 7 GeV mass for scalar particles is required to create on-shell tau-lepton pairs.
Feynman diagram of the scalar particle production mechanism is depicted in Figure (to be made soon).

Most CMS searches are not optimal for detecting Higgs boson decays to displaced jets
due to the soft $p_T$ nature of its decays products.
%Standard CMS searches rely on $H_T$ triggers that are highly inefficienty for this signal.
Low hadronic activity in transverse signature becomes particularly more difficult with a long-lived signature.
Higgs produced in association with Z vector boson analysis~\cite{ZHAN} overcame this barrier with help of dilepton trigger. 
Although ggH production mode gives the largest Higgs crosssection, it complicates the trigger strategy even further. 
This analysis exploits the $\tau$ lepton's leptonic decay, in which the $\tau$ lepton decays into a soft muon, using a trigger of B Parking High Level Trigger (HLT) Path implemented in CMS for the 2018 portion of Run 2.

Another challenge for a $\tau$ lepton analysis is on the different decay modes of $\tau$ leptons. 
$\tau$ leptons decay hadronically and leptonically, with several different sub-decay modes. 
Developing analysis strategies to optimize the search for each sub-decay modes is extremely complicated, a main reason for the omission or no good exclusion limit in precedent LHC results.
To be inclusive of all $\tau$ leptons' decay modes, a displaced vertex search can be more efficient than a displaced objects (jet, muon, electrons) search. 
We exploit the newly developed Regions of Interest mechanism in the tracker volume. 
Regions of Interest (ROI) form displaced vertex candidates, by fittng pair-wise tracks of Lost-tracks and PackedPFCandidates classes in MINIAOD data into a vertex. 
ROIs save all relevant track and fitted vertex qualities along with isolation information.
These variables are used as input for Machine Learning (ML) algorithms, enabling a highly generic and data-scientific search method.


The rest of the dissertation is organized as follows.
In Section~\ref{sec:theory}, we discuss the theoretical background for the need of the BSM theory in more details. 
In Section~\ref{sec:detectors}, the CMS detector is thoroughly discussed with emphasis on the tracks and the calorimeter, which are relevant detector parts for the analysis. 
We discuss how the analysis exploited the b-parking trigger in Section~\ref{sec:triggers}, with description of its original motive for the trigger's implementation. 
The physics objects and formation of Regions of Interest are described in Section~\ref{sec:objects}.
The machine learning algorithms are futher explained in Section~\ref{sec:machinelearning}. 
The event selections are presented in Section~\ref{sec:selections}. 
Section~\ref{sec:estimate} describes the data driven background estimation method. 
Section~\ref{sec:systs} describes the background estimation method's validation process and systematic uncertainties.
Finally, Section~\ref{sec:results} presents the results of the search.
We conclude with Section~\ref{sec:conclusions}.
