\clearpage
\chapter{Systematic Uncertainties}\label{sec:systs}
Table \ref{tab:systab} shows a summary of the systematic uncertainties included for the exclusion limit calculation,

\begin{table}[htb!]
  \caption{Systematic Uncertainty Table}
  \begin{center}
    \begin{tabular}{l|l|l}\hline
     Uncertainty    & Signal (\%) & Background (\%)\\
      \hline
     Luminosity     & 2 & 2\\
      \hline
     Muon ID,ISO Scale Factor     & 2 & 2 \\
      \hline
     ROI Score      & 10 & -\\
      \hline
     Background Estimation Method     & 20 & - \\
      \hline
     Statistical error     & 1-10 & 30 \\
      \hline
    \end{tabular}
    \label{tab:systab}
  \end{center}
\end{table}
CMS records the delivered luminosity, which is accompanied by its uncertainty.
This uncertainty is universal and applied to every analysis.
For data collected in 2018, the luminosity uncertainty is found to be 1.8\%, which is applied to both signal and background \cite{lumiUnc18}.
The analysis uses muon objects applied with Loose ID and ISO criteria.
Lepton ID, ISO scale factors, provided by the muon POG \cite{muonpog}, also carry uncertainty, albeit the magnitude is minimal.
The ROI score uncertainty is obtained by adjusting the leading ROI's vertex score.
As discussed in chapter \ref{sec:estimate}, we validated the ABCD method using a validation region of A'-D'.
The correlation factor of 2 variables in A'-D' was a small value of 11\%.
It tells us there could be an 11\% or slightly larger correlation in the A-D region.
A positive correlation implies more background concentrated in the SR(A) than its estimated rate(B*C/D), which would be about 11\%.
To be conservative with the estimation method, we set uncertainty sourced by the background estimation method at 20\%.
Last, our signal MC in SR(A) has enough statistics.
However, it is not infinite.
Therefore, we need to consider the statistical error of the signal MC in the signal region.
We have 100 events in signal region A, which results in 10\% of statistical error from $\frac{\sqrt{N}}{N}$, $\frac{10}{100}$.
Likely the background rate in the SR(A) is calculated with B*C/D.
Its propagated statistical uncertainty is @@@\%
