\clearpage
\chapter{B Parking Trigger Strategy}\label{sec:triggers}
The analysis signal process contains displaced SM $\tau$ leptons in their final state.
To exploit the leptonic decay of $\tau$ lepton with significant IP, specifically with the final muon state for a clean signal, the B-Parking triggers are used.
CMS implemented the B-Parking trigger in 2018 of Run 2 to research lepton universalities.
As described in chapter \ref{sec:detectors}, lepton universality tests claim that interaction between leptons and a gauge boson measures the same for each lepton.
In mathematical expression, R(K$^{*}$,D$^{*}$) defined in formulae \ref{eq:rk} and \ref{eq:rd}, are tested.
\begin{equation}
\label{eq:rk}
	R(K^{*})  = \frac{B->K^{*}\mu^{+}\mu^{-}}{B->K^{*}e^{+}e^{-}} 
\end{equation}
\begin{equation}
\label{eq:rd}
	R(D^{*})  = \frac{B->D^{*}\tau^{\pm}\nu}{B->D^{*}\mu^{\pm}\nu} 
\end{equation}

The first ratio specifically attracts the investigation of many physicists.
R(K$^{*}$)'s physics process is highly suppressed because Flavor-Changing Neutral Current (FCNC) is not allowed in the SM.
For a b-quark to change its flavor to a strange quark, it has to go through a loop process with two additional vertices, suppressing the cross-section.
Di-muon and di-electron signature of the highly suppressed process is very clean, making it an optimal channel to test lepton universality.
Its Feynman diagrams are shown in figures of \ref{fig:LU1} and \ref{fig:LU2}
\begin{figure}[h!]
	\caption{The figures show two different loop diagrams for B$->K^{*}l^{+}l^{-}$ processes. The FCNC is forbidden in the SM and there is no tree level process for B$->K^{*}l^{+}l^{-}$. Thus, these two processes are the leading contributors for B$->K^{*}l^{+}l^{-}$, which are highly suppressed.\cite{Lep:2017aai}}
  \label{fig:LU1}
  \centering
  \includegraphics[width=0.57\linewidth]{figs/Fig1a.png}
  \includegraphics[width=0.57\linewidth]{figs/Fig1b.png}
\end{figure}
\begin{figure}[h!]
	\caption{If the Lepton universality is not satisfied (R(K$^{*}$,D$^{*} \neq$ 1), it implies there is new physics hidden in the diagram. It could be interpreted in terms of Lepto-quark or Electroweak Z boson which couples to right-hand chrial leptons.\cite{Lep:2017aai}}
  \label{fig:LU2}
  \centering
  \includegraphics[width=0.57\linewidth]{figs/Fig1c.png}
  \includegraphics[width=0.57\linewidth]{figs/Fig1d.png}
\end{figure}
Because of this physics reason, the muon final state of B mesons are desired for research of R(K$^{*}$, D$^{*}$).
Consequently, the B-parking trigger requires a soft muon with modest displacement (measured using impact parameter) from the primary vertex, as in b-tagging.
It requires a muon with transverse momentum (pT) of 7-12 GeV with impact parameter (IP) significance 3-6.

Pp collisions in LHC produce enormous events, which could trigger the B parking trigger paths.
 As discussed in chapter \ref{sec:detectors}, the Current CPU capacity of CMS is limited and not capable of reconstructing the entire event at such a high trigger rate at the HLT level.
Thus, CMS scouts events, meaning it writes events that passed L1 trigger to a temporary dataset. 
Later, full HLT and RECO steps are implemented and serve as a B-Parking dataset.
The prescale factor for BPH triggers is 5-6.

\section{Trigger Paths}
We use data collected by the B-Parking triggers for the year of 2018.
The exact names of paths for B-Parking triggers are listed in Table~\ref{tab:triggers18}.
To compare with these data obtained from CMS detector, we get the Monte Carlo (MC) datasets for simulation of signal and background events.
Monte Carlo methods are computational algorithms to obtain physics results with statistical randomness.
CMS group uses this statistical instrument to publish various signal and background physics process MC datasets.
Using the generated signal MC, We plotted distributions of the generator level LLP's physics variables.
One can gauge triggering muons' IP significance values, pt, and isolation information.
\begin{table}[htb]
\caption{HLT trigger paths used in the analysis}
\begin{center}
\begin{tabular}{r|l}\hline
\hline
 Data sample & Trigger \\
\hline
 ParkingBPH*-Run2018A & HLT\_Mu9\_IP6\_part* \\
 ParkingBPH*-Run2018B & HLT\_Mu9\_IP6\_part* \\
 \hline
 ParkingBPH*-Run2018C & HLT\_Mu12\_IP6\_part* \\
 ParkingBPH*-Run2018D & HLT\_Mu12\_IP6\_part* \\
 \hline
 \hline
\end{tabular}
\label{tab:triggers18}
\end{center}
\end{table}

\begin{table}[htb]
\caption{Data and MC Global tags used for 2018 datasets}
\begin{center}
\begin{tabular}{r|l}\hline
 Data 2018 & 106X\_dataRun2\_v29 \\
 \hline
 MC 2018   & 106X\_upgrade2018\_realistic\_v11\_L1v1 \\
 \hline
\end{tabular}
\label{tab:GT}
\end{center}
\end{table}



\begin{figure}[h!]
  \caption{pt of the scalar products}
  \label{fig:scalarpt}
  \centering
  \includegraphics[width=0.5\linewidth]{figs/Scalar_pT100mm.pdf}
\end{figure}

\begin{figure}[h!]
  \caption{DeltaR of the scalar products}
  \label{fig:scalarpt}
  \centering
  \includegraphics[width=0.5\linewidth]{figs/Scalar_dR100mm.pdf}
\end{figure}

\begin{figure}[h!]
  \caption{liftime of the scalar products in the lab frame}
  \label{fig:scalarpt}
  \centering
  \includegraphics[width=0.57\linewidth]{figs/Scalar_gammactau100mm.pdf}
  \includegraphics[width=0.57\linewidth]{figs/Scalar_gammactau15GeV.pdf}
\end{figure}

Below is the trigger efficiency of various BPH trigger paths for different mass scale and lifetime points of the signal ($H \to SS \to \tau\tau\tau\tau$) sample
The signal process shows an overall good efficiency.
Signal points with LLP's c$\tau$ = 10,100mm show the best performance.
Signal points with LLP's c$\tau$ = 1000mm likely decay outside of the tracker region, leaving no track's impact parameter, and fails to pass the trigger.
On the other hand signal point with LLP's c$\tau$ = 1mm may not reach the first-pixel detector, which is at 2.7cm from the beam spot, and fails to pass the trigger.
We can confirm this explanation by observing that lighter LLP has better trigger efficiency for a shorter lifetime thanks to a more significant boost and vice versa for heavier LLP.
\begin{figure}[h!]
	\caption{The plots show the trigger efficiency for each HLT path with respect to LLP's lifetime. Each line denotes mass scale of each LLP. Please note the efficiency is set to 0 for MS=7GeV c$\tau$ = 1mm due to absece of Monte Carlo (MC).}
  \label{fig:Trigger Efficiency}
  \centering
  \includegraphics[width=0.47\linewidth]{figs/TrigEff_HLT_Mu7_IP4_part.pdf}
  \includegraphics[width=0.47\linewidth]{figs/TrigEff_HLT_Mu8_IP3_part.pdf}

\end{figure}
\begin{figure}[h!]
\caption{The plots show the trigger efficiency for each HLT path with respect to LLP's lifetime. Each line denotes mass scale of each LLP. The analysis uses Mu9\_IP6 for Era A,B of data and Mu12\_IP6 for Era C,D of data. Please note the efficiency is set to 0 for MS=7GeV c$\tau$ = 1mm due to absece of Monte Carlo (MC).}
  \centering
  \includegraphics[width=0.47\linewidth]{figs/TrigEff_HLT_Mu9_IP6_part.pdf}
  \includegraphics[width=0.47\linewidth]{figs/TrigEff_HLT_Mu12_IP6_part.pdf}

\end{figure}

In contrast to the signal, the background processes show poor trigger efficiency.
TTJets, Single Tops, and QCD pass the trigger at low but non-negligible efficiency.
All these background processes have heavy flavor particles for their final state (b-quark or top quark).
From the trigger efficiency and enormous cross-section of the QCD process, we can infer that the QCD process will become a significant contribution to our background.
Many physics analyses groups across CMS use the QCD MC datasets.
Thanks to their popularity and need for heavy statistics, CMS group generates QCD MC datasets filtered to different groups' focuses.
We use QCD's Muon Enriched datasets to improve background events' statistics, which would pass the B Parking trigger.
CMS group names the dataset QCD-MuEnrichedPt5, with QCD's muon decay product passing a pT threshold of 5 GeV.
QCD-MuEnriched has better efficiency for higher Pt bin samples since higher Pt bin samples tend to have more b-jets for their final state.
Drell-Yan and W-Jet processes show a very poor trigger efficiency due to the absence of heavy flavor particles in their final state.


\section{Integrated Luminosity and pileup weight for the HLT path}
The integrated luminosity for each era has been summarized in table \ref{tab:datasample2018BPH} in Appendix A.
The information was obtained with commands in section \ref{sec:PU} of Appendix A.
The integrated Luminosity totals at 44$fb^{-1}$ lower than 58.7$fb^{-1}$ for the year of 2018.
The bunch-crossing for the B-parking HLT path is also very different from other HLT paths.
As expected, b-parking data are recorded during lower bunch-crossing runs due to its extreme rate in CMS collider.

It is vital to adjust this bunch-crossing variable for MC simulation to model the data correctly.
To achieve this purpose, we apply pileup weight to the MC simulation.
Pileup weight is simply a bunch-crossing of data divided by the bunch-crossing of MC for a specific era.
Pileup (PU) weight values are calculated for each era of data-taking (A, B, C, D).
Figure~\ref{fig:EraAData} shows the BPH1-Era A's HLT\_Mu9\_IP6 HLT path's Data PU distribution.
\begin{figure}[h!]
  \caption{Bunch crossing of dataset /ParkingBPH1/Run2018A-UL2018\_MiniAODv2-v1/MINIAOD}
  \label{fig:EraAData}
  \centering
  \includegraphics[width=0.67\linewidth]{figs/NVtx_BPHA.pdf}

\end{figure}

Figure~\ref{fig:EraAData} shows the BPH1-Era B's HLT\_Mu9\_IP6 HLT path's Data PU distribution.
\begin{figure}[h!]
  \caption{Bunch crossing of dataset /ParkingBPH1/Run2018B-UL2018\_MiniAODv2-v1/MINIAOD}
  \label{fig:EraAData}
  \centering
  \includegraphics[width=0.67\linewidth]{figs/NVtx_BPHB.pdf}

\end{figure}

Figure~\ref{fig:EraCData} shows the BPH1-Era C's HLT\_Mu12\_IP6 HLT path's Data PU distribution.
\begin{figure}[h!]
\caption{Bunch crossing of dataset /ParkingBPH1/Run2018C-UL2018\_MiniAODv2-v1/MINIAOD. Please note that the HLT path for EraC has higher muon object's pT threshold with 12GeV (compared to 9GeV in EraA).}
  \label{fig:EraCData}
  \centering
  \includegraphics[width=0.57\linewidth]{figs/NVtx_BPHC.pdf}

\end{figure}

Figure~\ref{fig:EraCData} shows the BPH1-Era D's HLT\_Mu12\_IP6 HLT path's Data PU distribution.
\begin{figure}[h!]
\caption{Bunch crossing of dataset /ParkingBPH1/Run2018D-UL2018\_MiniAODv2-v1/MINIAOD. Please note that the HLT path for EraD has higher muon object's pT threshold with 12GeV (compared to 9GeV in EraA).}
  \label{fig:EraCData}
  \centering
  \includegraphics[width=0.57\linewidth]{figs/NVtx_BPHD.pdf}

\end{figure}

Figure~\ref{fig:MCPU} shows DYJetsToLL\_M-50\_TuneCP5\_13TeV-madgraphMLM-pythia8 MC PU distribution.
\begin{figure}[h!]
	\caption{Bunch crossing of Monte Carlo Simulation for physics process of Drell-Yan. The dataset is /DYJetsToLL\_M-50\_TuneCP5\_13TeV-madgraphMLM-pythia8
\newline/RunIISummer20UL18MiniAODv2-106X\_upgrade2018\_realistic\_v16\_L1v1-v2/\newline MINIAODSIM}
  \label{fig:MCPU}
  \centering
  \includegraphics[width=0.67\linewidth]{figs/NVtx_DYJetsToLL_M-50_try.pdf}

\end{figure}

Figure~\ref{fig:PUWeight9} shows resultant such PUWeight from BPH1\_A HLT\_Mu9\_IP6 and DYJetsToLL\_M-50\_TuneCP5\_13TeV-madgraphMLM-pythia8.
\begin{figure}[h!]
  \caption{PUweight calculated for Era A of B-parking dataset}
  \label{fig:PUWeight9}
  \centering
  \includegraphics[width=0.67\linewidth]{figs/NVtx_PUWeight.pdf}

\end{figure}
