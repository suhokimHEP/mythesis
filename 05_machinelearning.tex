\clearpage
\chapter{Machine Learning}\label{sec:machinelearning}

ROIs are artificial regions created by the CMSSW mechanism, which are displaced vertex candiates.
ROIs contain thorough information about fitted tracks, vertices, and isolation information, after following the formation procedure described in the previous section. 
The exhaustive variables saved in each ROI contain information, which directly and indirectly tells whether the ROI is from signal process or background process.
Given the extensive amount of variables within ROIs, it's inappropriate to use ROIs' single or a few variabes as our tagging variables, like in ZH analysis ~\cite{ZHAN}.  
It is also ineffcient to implement a cut-based approach due to having approximately 20-30 variables from ROI.
Optimization process for all 20-30 variables would be extremley time-consuming, inefficient, and error-prone as well.
Therefore, the analysis exploits machine learning (ML) for these multivariate analysis.
Boosted decision trees will also face similar problem as cut-based analysis, although to a lesser extent.
Deep Neural Network are the most adaptable ML algorithm for this analysis method, in which the algorithm inputs extensive list of ROI variables into a neural network, and receives a final score to discriminate ROIs that arise in signal process versus background processes. 
The list of variables that are inputted into the ML algorithms are provided in subsection \ref{sec:MLIV}.

The analysis uses Keras-Tensorflow for its ML platform.
CMSSW includes Keras-Tensorflow, which enables simple and easy usage of keras-Tensorflow with simple cmsenv command in various CMS remote clusters. 
For CMSSW\_10\_6\_12 (to have access for BPH trigger path), Keras-Tensorflow version is 2.3.1.
It runs with CPUs and GPUs.

\section{Machine Learning Input Variables}\label{sec:MLIV}
As discussed in chapter \ref{sec:objects}, the ROI has vertex and shell (isolation) information.
Vertex information is trained with 19 different variables, which are listed and categorized in Tables ~\ref{tab:ROITCvars}.
%, ~\ref{tab:ROIANvars}, and ~\ref{tab:ROIEVvars}.

\begin{table}[htb]
\caption{ROI (trackCluster) variables by category}
\begin{center}
\begin{tabular}{r|l|l}\hline
 TrackCluster & Position & TrackClusters.vx() - primaryVertex.X() \\
              & Position & TrackClusters.vy() - primaryVertex.Y() \\
              & Position & TrackClusters.vz() - primaryVertex.Z() \\
              & Covariance & TrackClusters.vertexCovariance()(0,0) \\
              & Covariance & TrackClusters.vertexCovariance()(0,1) \\
              & Covariance & TrackClusters.vertexCovariance()(0,2) \\
              & Covariance & TrackClusters.vertexCovariance()(1,0) \\
              & Covariance & TrackClusters.vertexCovariance()(1,1) \\
              & Covariance & TrackClusters.vertexCovariance()(1,2) \\
              & Track0,1 & Track0,1.pt \\
              & Track0,1 & Track0,1.eta \\
              & Track0,1 & Track0,1.phi \\
              & Track0,1 & Track0,1.dxy \\
              & Track0,1 & Track0,1.dz \\
              & Track0,1 & Track0,1.normalizedChi2 \\
              & Track0,1 & Track0,1.HighPurityInt \\
 \hline
 \hline
\end{tabular}
\label{tab:ROITCvars}
\end{center}
\end{table}



Shell information is trained with 8 different variables, which are listed in Tables ~\ref{tab:ROIANvars}.

\begin{table}[htb]
\caption{ROI (Annulus) variables by category}
\begin{center}
\begin{tabular}{r|l|l}\hline
 Annulus      & pfCandidate/LostTracks & pfCandidate/LostTracks.pt \\
              & pfCandidate/LostTracks & pfCandidate/LostTracks.eta \\
              & pfCandidate/LostTracks & pfCandidate/LostTracks.phi \\
              & pfCandidate/LostTracks & pfCandidate/LostTracks.dxy \\
              & pfCandidate/LostTracks & pfCandidate/LostTracks.dz \\
              & pfCandidate/LostTracks & pfCandidate/LostTracks.normalizedChi2 \\
              & pfCandidate/LostTracks & pfCandidate/LostTracks.HighPurityInt \\
              & pfCandidate/LostTracks & pfCandidate/LostTracks.DeltaR(trackMomentum) \\
 \hline
 \hline
\end{tabular}
\label{tab:ROIANvars}
\end{center}
\end{table}

We input vertex and shell information into separate ML algorithms. Thus, we get 2 final products from the ML algorithms, one for the vertex and the other for the shell.
%\begin{table}[htb]
%\caption{Event variables by category}
%\begin{center}
%\begin{tabular}{r|l|l}\hline
%  ROI    & Position & x \\
%         & Position & y \\
%         & Position & z \\
% \hline
% \hline
%\end{tabular}
%\label{tab:ROIEVvars}
%\end{center}
%\end{table}

\subsection{Fine-Tuning of ML environemnt}
CMSSW includes image container for Keras-Tensorflow and researchers can submit remote batch jobs for its ML training as well.
The analysis tested multiple variables (such as epoch numbers, batch sizes, phi sizes, f sizes) of our DNN layers thanks to submission of remote batch jobs with CMSSW's Keras-Tensorflow image container.
Subsection ~\ref{sec:Ethan} below details the numerical results from such tests.
The analysis also tested different combinations of signal vs background datasets for maximum discriminant power. 
The list of mass scale and lifetime tested for signal points are listed in subsection~\ref{sec:Ethan2}. Different SM physics process (and their compositions) tested for background process are also listed in subsection~\ref{sec:Ethan2}

The final Tensorflow product, which was used for the analysis, were trained with parameters in the following section. 
With these variables, Keras-Tensorflow DNN variable information was fine-tuned for maximum Area Under the Curve (AUC) calue.
Its information is listed in the table \ref{tab:ROIParam}.
\begin{table}[htb]
\caption{Tensorflow information}
\begin{center}
\begin{tabular}{r|l}\hline
Epoch & 300 \\
batch size & 250 \\
Phi sizes & (64,128,256) for vertex ,(32,64,128) for shell \\
f sizes & (256,128,32) \\
Signal & ggHSSTo4Tau-MS15GeV-c$\tau$100mm  \\
Background & QCD\_Pt120-170\_MuEneriched and TTJets \\
 \hline
 \hline
\end{tabular}
\label{tab:ROIParam}
\end{center}
\end{table}
\subsection{DNN Variable Test}\label{sec:Ethan}
Batch size 


\subsection{Signal and Background MC Test}\label{sec:Ethan2}
For Ethan.

\begin{table}[htb]
\caption{Tensorflow information}
\begin{center}
\begin{tabular}{r|l|l|l|l|l}\hline
Epoch &  loss & acc &  val\_loss & val\_acc & AUC\\
\hline
100& 0.2161 & 0.9074 & 0.2469 & 0.8942 & 0.9387\\
150& 0.2039 & 0.9120 & 0.2393 & 0.8983 & 0.9414\\
200& 0.1934 & 0.9151 & 0.2523 & 0.8953 & 0.9408\\
250& 0.1977 & 0.9144 & 0.2459 & 0.8982 & 0.9399\\
300& 0.1738 & 0.9272 & 0.2573 & 0.8977 & 0.9387\\
350& 0.1607 & 0.9332 & 0.2693 & 0.8934 & 0.9403\\
400& 0.1459 & 0.9387 & 0.2823 & 0.8970 & 0.9394\\
\hline
\end{tabular}
\label{tab:Epoch Training}
\end{center}
\end{table}

\begin{table}[htb]
\caption{Tensorflow information}
\begin{center}
\begin{tabular}{r|l|l|l|l}\hline
Signal & Background & Loss & Accuracy & AUC\\
\hline
ggH\_HToSSTo4Tau\_MH-125\_MS-7\_ctauS-10& QCDMuEnrichedPt5\_Pt20-30 & 0.1852 & 0.9244 & 0.9696\\
& QCDMuEnrichedPt5\_Pt470-600 & 0.1597 & 0.9361 & 0.9727\\
& TTJets & 0.1681 & 0.9320 & 0.9680\\
\hline
ggH\_HToSSTo4Tau\_MH-125\_MS-15\_ctauS-10& QCDMuEnrichedPt5\_Pt20-30 & 0.2133 & 0.9093 & 0.9598\\
& QCDMuEnrichedPt5\_Pt470-600 & 0.1687 & 0.9306 & 0.9679\\
& TTJets & 0.1838 & 0.9242 & 0.9610\\
\hline
ggH\_HToSSTo4Tau\_MH-125\_MS-15\_ctauS-100& QCDMuEnrichedPt5\_Pt20-30 & 0.0747 & 0.9755 & 0.9791\\
& QCDMuEnrichedPt5\_Pt470-600 & 0.1596 & 0.9361 & 0.9727\\
& TTJets & 0.1681 & 0.9320 & 0.9680\\
\hline
ggH\_HToSSTo4Tau\_MH-125\_MS-40\_ctauS-100& QCDMuEnrichedPt5\_Pt20-30 & 0.1898 & 0.9209 & 0.9695\\
& QCDMuEnrichedPt5\_Pt470-600 & 0.1765 & 0.9275 & 0.9698\\
& TTJets & 0.1576 & 0.9379 & 0.9705\\
\hline
ggH\_HToSSTo4Tau\_MH-125\_MS-55\_ctauS-100& QCDMuEnrichedPt5\_Pt20-30 & 0.1898 & 0.9227 & 0.9635\\
& QCDMuEnrichedPt5\_Pt470-600 & 0.1480 & 0.9399 & 0.9674\\
& TTJets & 0.1433 & 0.9450 & 0.9706\\
\hline
\end{tabular}
\label{tab:Epoch Training}
\end{center}
\end{table}

\section{ML scores of the ROIs}

ROIs of all events inside the dataset listed in section ~\ref{sec:samples} are scored with the tensorflow, when the tensorflow was trained with information from section ~\ref{sec:MLIV}.
Because the signal process has 2 scalar decays as in Figure~\ref{fig:feynmanggH}, it's reasonable to require 2 high-scoring ROIs for our analysis.
However, the 2 selected ROIs are not simply the 2 highest-scoring ROIs of the event, due to non-negligible lifetime of $\tau$ leptons (from signal process in the detector.
After obtaining an ROI with highest ML output socre, we search for the next highest scored ROI that are $\Delta\Phi$>0.4 from the leading ROI.
Vertex discriminant is more powerful thatn the shell discriminant.
Thus, the ordering of ROIs scores are only done by the vertex ML output value.
%More detailed explanations are described in section~\ref{sec:lifetimeROI}.)  
%The cuts on the leading ROI, and the subleading ROI are optimized based on maximizing the punzi significance formula with value $\sigma$ set to discovery value of 5.  
%\begin{equation}\label{eq:significance}
%    \sigma(N_{\mathrm{displaced-tag}}) = \frac{S(N_{\mathrm{displaced-tag}})} {\sqrt{B(N_{\mathrm{displaced-tag}})} + 2.5 }.
%\end{equation}

\begin{figure}[h!]
  \caption{Tensorflow scores}
  \label{fig:TensorFlow scores}
  \centering
  \includegraphics[width=0.67\linewidth]{figs/Tensorflow_Disc_mostrecent.pdf}

\end{figure}

 \begin{figure}[h!]
   \caption{Signal versus Background for log10(1-ROIscore), where the ROI score is the highest ROI of the event. Left plot is for MS-15\_ctauS-10mm point, whereas the right plot is for MS-15\_cauS-100mm point}
   \label{fig:leadROIscore}
   \centering
   \includegraphics[width=0.47\linewidth]{figs/AnalysisNoteplot_MS-15_ctauS-10_hloglead.pdf}
   \includegraphics[width=0.47\linewidth]{figs/AnalysisNoteplot_MS-15_ctauS-100_hloglead.pdf}
 \end{figure}


 \begin{figure}[h!]
   \caption{Signal versus Background for log10(1-subROIscore), where the ROI score is the second highest ROI (outside of dPhi=0.4 from leading ROI) of the event. Left plot is for MS-15\_ctauS-10mm point, whereas the right plot is for MS-15\_cauS-100mm point}
   \label{fig:excROIscore}
   \centering
   \includegraphics[width=0.47\linewidth]{figs/AnalysisNoteplot_MS-15_ctauS-10_hlogexclead.pdf}
   \includegraphics[width=0.47\linewidth]{figs/AnalysisNoteplot_MS-15_ctauS-100_hlogexclead.pdf}
 \end{figure}


\begin{figure}[h!]
  \caption{Data/MC agreement for ROI scores}
  \label{fig:DataMCscore}
  \centering
  \includegraphics[width=0.67\linewidth]{figs/Data_AnalysisNoteplot_MS-15_ctauS-10_ROIscore.pdf}

\end{figure}


\begin{figure}[h!]
  \caption{Data/MC agreement for loglead/sublead scroes}
  \label{fig:DataMCscore}
  \centering
  \includegraphics[width=0.47\linewidth]{figs/Data_AnalysisNoteplot_MS-15_ctauS-10_hloglead.pdf}
  \includegraphics[width=0.47\linewidth]{figs/Data_AnalysisNoteplot_MS-15_ctauS-10_hlogexclead.pdf}
\end{figure}



