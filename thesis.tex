\documentclass[11pt,expanded,copyright]{fsuthesis}

%\usepackage{lmodern}
\usepackage{amsmath}
\usepackage{amssymb}
\usepackage{caption}
\usepackage{subcaption}
\usepackage{lscape}
\usepackage{mflogo}         % (not needed for general use)
\usepackage{multicol}
\usepackage{xspace}
\usepackage{upgreek}
\usepackage[overload]{textcase}

\usepackage[american]{babel}
\usepackage{csquotes}
\usepackage[sorting=none]{biblatex}
%\usepackage[backend=biber]{biblatex}
\addbibresource{myrefs.bib}

\addto\captionsamerican{\renewcommand*{\contentsname}{Table of Contents}}

\ifpdf   % We execute this part (up to \else) if we are in PDF mode.
  \usepackage[pdftex]{graphicx}
  \usepackage{color}
  \definecolor{mygreen}{rgb}{0,0.6,0}
  \definecolor{myblue}{rgb}{0.3,0.2,0.8}
  \definecolor{myred}{rgb}{0.8,0.1,0.1}
  \usepackage[colorlinks=true,bookmarks=true,pdfborder={0 0 0},
    linkcolor=myblue,urlcolor=myred,citecolor=mygreen,
    breaklinks=true,bookmarksnumbered=true]{hyperref}
\else    % Otherwise we'll run this part if we are not in PDF mode.
  \usepackage[dvips]{graphicx}
\fi

\usepackage{lipsum}

\widowpenalty=9999
\clubpenalty=9999

%  Experiments
\newcommand {\DZERO}{D0\xspace}     %etc.


% Measurements and units...

\newcommand{\GeV}{\ensuremath{\,\text{Ge\hspace{-.08em}V}}\xspace}
\newcommand{\GeVns}{\ensuremath{\text{Ge\hspace{-.08em}V}}\xspace} % no leading thinspace
\newcommand{\gev}{\GeV}
\newcommand{\TeV}{\ensuremath{\,\text{Te\hspace{-.08em}V}}\xspace}
\newcommand{\TeVns}{\ensuremath{\text{Te\hspace{-.08em}V}}\xspace} % no leading thinspace
\newcommand{\lumi}{\ensuremath{\mathcal{L}}\xspace}
\newcommand{\Lumi}{\ensuremath{\mathcal{L}}\xspace}%both upper and lower
\newcommand{\Mpl}{\ensuremath{{M_\mathrm{Pl}}}\xspace}% Planck mass
\newcommand{\mm}{\ensuremath{\,\text{mm}}\xspace}
%

% Physics symbols ...

\newcommand{\PT}{\ensuremath{p_{\mathrm{T}}}\xspace}
\newcommand{\pt}{\ensuremath{p_{\mathrm{T}}}\xspace}
\newcommand{\ET}{\ensuremath{E_{\mathrm{T}}}\xspace}
\newcommand{\HT}{\ensuremath{H_{\mathrm{T}}}\xspace}
\newcommand{\mT}{\ensuremath{m_{\mathrm{T}}}\xspace}
\newcommand{\mTii}{\ensuremath{m_{\mathrm{T2}}}\xspace}
\newcommand{\et}{\ensuremath{E_{\mathrm{T}}}\xspace}
\newcommand{\Em}{\ensuremath{E\hspace{-0.6em}/}\xspace}
\newcommand{\Pm}{\ensuremath{p\hspace{-0.5em}/}\xspace}
\newcommand{\PTm}{\ensuremath{{p}_\mathrm{T}\hspace{-1.02em}/\kern 0.5em}\xspace}
\newcommand{\PTslash}{\PTm}
\newcommand{\ETm}{\ensuremath{E_{\mathrm{T}}^{\text{miss}}}\xspace}
\newcommand{\MET}{\ETm}
\newcommand{\ETmiss}{\ETm}
\newcommand{\ptmiss}{\ensuremath{\pt^\text{miss}}\xspace}
\newcommand{\ETslash}{\ensuremath{E_{\mathrm{T}}\hspace{-1.1em}/\kern0.45em}\xspace}
\newcommand{\VEtmiss}{\ensuremath{{\vec E}_{\mathrm{T}}^{\text{miss}}}\xspace}
\newcommand{\ptvec}{\ensuremath{{\vec p}_{\mathrm{T}}}\xspace}
\newcommand{\ptvecmiss}{\ensuremath{{\vec p}_{\mathrm{T}}^{\kern1pt\text{miss}}}\xspace}
\newcommand{\tauh}{\ensuremath{\PGt_\mathrm{h}}\xspace}
\newcommand{\sqrtsNN}{\ensuremath{\sqrt{\smash[b]{s_{_{\mathrm{NN}}}}}}\xspace}
\newcommand{\mht}{\ensuremath{H_{\mathrm{T}}^{\text{miss}}}\xspace}
\newcommand{\htvecmiss}{\ensuremath{\vec{H}_{\text{T}}^{\text{miss}}}\xspace}

\newcommand{\cmsSymbolFace}[1]{#1}
%\newcommand{\cmsSymbolFace}[1]{\mathrm{#1}}
%\ifthenelse{\boolean{cms@italic}}{\newcommand{\cmsSymbolFace}[1]{#1}}{\newcommand{\cmsSymbolFace}[1]{\mathrm{#1}}}
% future symbols from heppennames2
\providecommand{\PH}{\ensuremath{\cmsSymbolFace{H}}\xspace} % plain Higgs
\providecommand{\PJGy}{\ensuremath{\cmsSymbolFace{J}\hspace{-.08em}/\hspace{-.14em}\psi}\xspace} % J/Psi (no mass)
\providecommand{\PBzs}{\ensuremath{\cmsSymbolFace{B}^0_\cmsSymbolFace{s}}\xspace} % B^0_s
\providecommand{\Pg}{\ensuremath{\cmsSymbolFace{g}}\xspace} % generic gluon
\providecommand{\PSg}{\ensuremath{\widetilde{\cmsSymbolFace{g}}}\xspace} % gluino
\providecommand{\PSQ}{\ensuremath{\widetilde{\cmsSymbolFace{q}}}\xspace} % squark
\providecommand{\PXXG}{\ensuremath{\cmsSymbolFace{G}}\xspace} % graviton
\providecommand{\PXXSG}{\ensuremath{\widetilde{\PXXG}}\xspace} % gravitino
\providecommand{\PSGcp}{\ensuremath{\widetilde{\chi}^+}\xspace}
\providecommand{\PSGc}{\ensuremath{\widetilde{\chi}}\xspace} % neutralino
\providecommand{\PSGcz}{\ensuremath{\widetilde{\chi}^0}\xspace} % neutralino with superscript 0
\providecommand{\PSGczDo}{\ensuremath{\widetilde{\chi}^{0}_{1}}\xspace} % neutralino
\providecommand{\PSGczDt}{\ensuremath{\widetilde{\chi}^{0}_{2}}\xspace} % neutralino
\providecommand{\PSGcpm}{\ensuremath{\widetilde{\chi}^\pm}\xspace} % neutralino
\providecommand{\Pl}{\ensuremath{\cmsSymbolFace{l}}\xspace} % non-ell lepton
\providecommand{\PAl}{\ensuremath{\overline{\cmsSymbolFace{l}}}\xspace} % non-ell anti-lepton
\providecommand{\PGnl}{\ensuremath{\nu_\cmsSymbolFace{l}}\xspace} % lepton neutrino
\providecommand{\PAGnl}{\ensuremath{\overline{\nu}_\cmsSymbolFace{l}}\xspace} % anti-lepton neutrino
\providecommand{\PQtpr}{\ensuremath{\cmsSymbolFace{t}^{\prime}}\xspace} % t'
\providecommand{\PQt}{\ensuremath{\cmsSymbolFace{t}}\xspace} % t
\providecommand{\PQb}{\ensuremath{\cmsSymbolFace{b}}\xspace} % b
\providecommand{\PAQtpr}{\ensuremath{\bar{\cmsSymbolFace{t}}^\prime}\xspace} % t'-bar; needs to be converted to overline-requires rework a la heppennames
\providecommand{\PAQt}{\ensuremath{\bar{\cmsSymbolFace{t}}}\xspace} % t'-bar; needs to be converted to overline-requires rework a la heppennames
\providecommand{\PQbpr}{\ensuremath{\cmsSymbolFace{b}^{\prime}}\xspace} % b'
\providecommand{\PAQbpr}{\ensuremath{\bar{\cmsSymbolFace{b}}^\prime}\xspace} % b'-bar; needs same as anti-t'
\providecommand{\PGg}{\ensuremath{\gamma}\xspace} % gamma
\providecommand{\PKzS}{\ensuremath{\cmsSymbolFace{K}^0_\cmsSymbolFace{S}}\xspace} % K short
\providecommand{\PBs}{\ensuremath{\cmsSymbolFace{B}_\cmsSymbolFace{s}}\xspace} % B sub s
\providecommand{\PSQt}{\ensuremath{\widetilde{\cmsSymbolFace{t}}}\xspace} % stop
\providecommand{\PZpr}{\ensuremath{\cmsSymbolFace{Z}^\prime}} % plain Z'
% Particle names which track the italic/non-italic face convention
\newcommand{\zp}{{\PZpr}\xspace} % plain Z'
\newcommand{\JPsi}{{\PJGy}\xspace} % J/Psi (no mass)
\newcommand{\Z}{{\PZ}\xspace} % plain Z (no superscript 0)
\newcommand{\ttbar}{{\PQt{}\PAQt}\xspace} % t-tbar




\title{Search for Higgs boson decays to long-lived scalar particles with Regions of Interest and Machine Learning in CMS}
                                      %% Manuscript title
\author{Suho Kim}  %% Testing accented characters in name
\college{College of Arts and Sciences}     %% School or College
\department{Department of Physics}  %% Delete if no department
\manuscripttype{Dissertation}               %% [Thesis, Dissertation, Treatise]
\degree{Doctor of Philosophy}            %% Name of the degree
\degreeyear{2022}                     %% Graduation Yea7\defensedate{Nov 18, 2022}             %% Date of Defense
\defensedate{Nov 7, 2022}             %% Date of Defense
\subject{Dissertation Formatting}     %% PDF document metadata subject
\keywords{latex; fsuthesis; etd; tables; figures; bibtex; document formatting}
				      %% PDF document metadata search
                                      %%   keywords, separated by semicolons

\committeeperson{Ted Kolberg}{Professor Directing Disseration}
\committeeperson{Elizabeth Hammock}{University Representative}
\committeeperson{Todd Adams}{Committee Member}
\committeeperson{Kohsaku Tobioka}{Committee Member}
\committeeperson{Peter Hoeflich}{Committee Member}

\newcommand*{\acro}[1]{{\small\textsc{#1}}}
\newcommand*{\lit}[1]{\texttt{#1}}
\newcommand*{\pkg}[1]{\textsf{#1}}


\begin{document}

\frontmatter          %% Establish small roman numeral numbering
\maketitle            %% Create the title page
\makecommitteepage    %% Create the committee page

\begin{dedication}
\centering
To my parents, who always suspected I'd end up here
\end{dedication}

\begin{acknowledgments}
Thanks to many people.
Faculty
Collaboration members 
Pedro
Fifth floor freinds
Caleb
Class of 2016 friends
I want to thank my family members.
\end{acknowledgments}

\tableofcontents
\listoftables
\listoffigures
\begin{listofsymbols}
The following short list of symbols are used throughout the document.
The symbols represent quantities that I tried to use consistently.
The symbols follow CMS Technical Design Report (TDR) style.
\begin{center}
\begin{tabular}{ll}
\DZERO & D0 experiment \\
\GeV & 1 Giga eletron-Volt \\
\TeV & 1 Tera eletron-Volt \\
$\Lumi_{int}$ & Integrated luminosity\\
\pt & Transverse momentum \\
$\eta$ & Pseudorapidity \\
\ET & Transverse energy\\
\HT & Scalar sum of hadronic jet's transverse energy\\
\MET & Missing transverse energy\\
Pp & Proton-to-Proton collision\\
\PH & Higgs\\
$\pi^{0}$ & Pion (pi-zero)\\
$\PKzS$ & K short\\
\PJGy & J-Psi meson\\
$\PBzs$ & strange B meson\\
$\Upsilon$ & Upsilon meson\\
\Pg & Gluon\\
\PQb & b quark\\
\PQt & t quark\\
e & Electron lepton\\
$\mu$ & Muon lepton\\
$\tau$ & Tau lepton\\
j & Jet\\
\ttbar & t-tbar \\
\Mpl & Planck mass\\
MS & Mass scale of particle\\
c$\uptau$ & Lifetime of particle in its rest frame\\
$\ell$ & Electrically charged leptons\\
\PGnl & Generic neutrino\\
\end{tabular}
\end{center}
\end{listofsymbols}
%\listofmusex

%%% You may also create a list of abbreviations if it may be of use to
%%% your readers. Otherwise, this section may be deleted or remain
%%% commented. Instead, if you need another frontmatter environment
%%% for some other use (besides a list of abbreviations or symbols),
%%% you could rename the heading to suit your purposes. For example,
%%% if I need an Index of Scary Movies, I can steal the
%%% 'listofabbrevs' environment and rename its heading. (The
%%% environment name stays the same, but heading will be changed on
%%% the page and in the table of contents. The 'listofsymbols'
%%% environment above may also be co-opted in this way.) Here's how I
%%% might do this:

%\renewcommand*{\listabbrevname}{Index of Scary Movies}
%\begin{listofabbrevs}
%  [... insert scary movie index material here ...]
%\end{listofabbrevs} 

%%% Now I use the abstract environment to create an abstract page.
%%% The heading is created automatically.  The rest of the content
%%% is up to you.
\begin{abstract}
We present a search for long-lived particles (LLPs) produced in gluon fusion Higgs production mode (ggH), using a novel strategy of Regions of Interest (ROIs). Regions of Interest are collections of pair-wise track vertices fitted by the vertex fitter in CMSSW.
   The analysis focuses on LLPs with lifetimes of mm-m ranges, and result in decay of collections of pair-wise track vertices.
   Variables of the constructed ROIs become inputs for our Deep Neural Network (DNN) Machine Learning (ML) algorithms, as the main discriminator between the signal and the background.
	We look for LLPs that decay into two tau leptons. 
	This final state is particularly interesting, given tau lepton final state exclusion limits are mostly omitted in preceding analyses, due to the difficulty of reconstructing tau leptons.
  To trigger on this final state, we exploit the B-parking trigger.
 The B-parking trigger is a novel High-Level Trigger path in CMS detector, targetting soft displaced muons.
 The trigger was installed in 2018, totaling an integrated Luminosity value of $\Lumi_{int}$ = 44fb$^{-1}$.
  No excess of events over the standard model expectation is observed.
  The results are interpreted in the context of exotic Higgs decays to a pair of long-lived scalars ($S$).
	We set limits on the branching ratio of the Higgs to LLPs, \textbf{B}(\PH$\to SS \to \tau^{+}\tau^{-}\tau^{+}\tau^{-}$)
  , as a function of the proper lifetime.
  The analysis has the strongest discriminant power for the LLP's c$\uptau$ from 1\mm to 10\mm for 7 and 15\GeV, and from 10\mm to 100\mm for 40 and 55\GeV.
	The analysis' exclusion limit on the branching ratio for $\tau$ lepton final state (\textbf{B}(H$\rightarrow SS \rightarrow \tau^{+}\tau^{-}\tau^{+}\tau^{-}$)) is one of the most stringent results from the LHC detectors.
 We place limits on the Higgs branching ratio to this final state down to 9\% at 10\mm for 7\GeV, 5\% at 10\mm for 15\GeV, 5\% at 100\mm for 40\GeV, and 18\% at 100\mm for 55\GeV.
\end{abstract}

%%% The abstract is the last element of the so-called "front matter"
%%% of the document.  We now move on to the "main matter" --- the
%%% chapter material (and optional appendices).  By calling the
%%% \mainmatter macro, we reset page numbering to arabic numerals
%%% beginning with '1'.

\mainmatter

%%% I have chosen to create one new LaTeX file for each chapter of my
%%% document.  This decision is arbitrary.  I may break the document
%%% into as many or as few pieces as I like.  I may name the chapter
%%% and appendix files anything I like (though I might have some
%%% trouble if I use spaces or odd characters in my file names).  The
%%% only requirement is that the file name ends with the extension
%%% '.tex'.  For the \input lines below, I can leave out the '.tex'
%%% extension.  If instead I prefer to create one long file containing
%%% my entire thesis, I may just continue on here by starting with the
%%% first \chapter command of my document.  (In this case, I would not
%%% need the following '\input' commands at all.)  I have chosen to
%%% name my chapter files by the concepts they contain.

\input 00_introduction
\input 01_theory
\input 02_samples
\input 03_triggerstrategy
\input 04_objectreconstruction
\input 05_machinelearning
\input 06_eventselection
\input 07_bkgestimate
\input 08_systematics
\input 09_results
\input 10_conclusions
%%%\input math
%%%\input citations
%%%\input figures
%%%\input tables

%%% Having finished the main text of my document, I will move on to
%%% the appendices (or my one appendix, in this case).  An appendix
%%% will look exactly like a chapter, even including a \chapter macro
%%% call.  By first calling the \appendix macro, I will cause all
%%% subsequent chapter headings to be labeled "Appendix" and to be
%%% enumerated by letters rather than numbers.  I.e., the first
%%% appendix will be titled 'Appendix A', then 'Appendix B', and so
%%% on.  This also adds a heading entry to the Table of Contents
%%% labeled 'Appendix'.  If you have more than one appendix, you may
%%% want to change the ToC heading to read 'Appendices'.  If you need
%%% to do this, you may reset the value of '\appendixtocname' before
%%% the call to the \appendix macro:

%\renewcommand*{\appendixtocname}{Appendices} % For more than one appendix
\appendix
\input appendix


\renewcommand*{\bibname}{References}

%%% The skeleton document in the thesis-template folder gives an
%%% example of the "simple" 'references' environment.  In that
%%% document, you must format all of your own references and manage
%%% your own citation styling.  In contrast, this document assumes
%%% that I will be using BibLaTeX to format my references.  I'll be
%%% using the default plain format style, and the bibliographic data
%%% are kept in the file 'myrefs.bib'. This is specified in the
%%% document preamble above.

%%% If using biblatex/biber, then this is the command required
%%% to render the bibliography/references section correctly.

\printbibliography

%%% If you're using the older BibTeX system, then selecting the
%%% styling and the bibliography invocation are a little different.

%%% Plain LaTeX with BibTeX provides just a few citation and
%%% bibliography styles.  Many more styles are available using
%%% 'natbib', 'apacite', or other packages. A quick web search with
%%% 'bibtex' and your own discipline as keywords may help you locate
%%% both references and the appropriate style files for you to
%%% download. If you '\usepackage{natbib}' or '\usepackage{apacite}'
%%% at the top of this file, then you should uncomment the appropriate
%%% alternative style for the package you've chosen. Uncomment only
%%% one of these:

%\bibliographystyle{ieeetr}    % This is the default LaTeX/BibTeX style
%\bibliographystyle{plainnat} % If using 'natbib', use this line instead
%\bibliographystyle{apacite}  % If using 'apacite', use this line instead

%%% For BibTeX, the following line actually generates the bibliography
%%% using the data supplied in 'myrefs.bib', the citations in my
%%% document, and the style I've selected above.

%\bibliography{myrefs}

%%% The last element of the document is the biographical sketch.  The
%%% heading and table of contents entry are automatically created by
%%% using the biosketch environment.  The rest of the content is up to
%%% you.  Remember not to include any personal contact info.

\begin{biosketch}
The author was born, and then the author was ``educated,'' at least to
some degree.  After finishing high school, the author
completed a Bachelor of Science degree at Washington University in St.Louis.
Following a decade in the work force in his discipline, the author
went to FSU to pursue graduate work.
Born Education Passage upto PH.D
Research and seminar topics
Achievement bullet points
\end{biosketch}

%%% Here endeth the document.  LaTeX will ignore anything that follows
%%% the \end{document} command.

\end{document}
