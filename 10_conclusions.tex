\clearpage
\chapter{Conclusions}\label{sec:conclusions}
The SM still needs extra particles to solve the naturalness problem.
The theory of neutral naturalness, which includes Twin Higgs' models and others, postulates long-lived neutral scalar particles.
The SM Higgs boson can become a portal to the BSM LLP, and the SM Higgs boson can be created in CMS of the LHC.
We used CMS data to observe if we could detect LLPs via the Higgs Portal.
Preceding analyses already tested LLPs' decays into \PQb and d quarks, concluded CMS data could not detect LLPs via \PQb and d quark channels, and set limits on the branching ratio of the Higgs to LLPs.
However, preceding analyses were unsuccessful in LLP's $\tau$ channel decay due to the difficulty in trigger and reconstruction strategy.
The analysis exploited a novel B-Parking HLT path to trigger the LLP's $\tau$ channel decay while maximizing the signal process's cross-section by targeting the ggH production mode.
We reconstructed Regions of Interest, a purely tracker-based object, to avoid subdividing $\tau$ lepton's multiple decay modes.
Signal ROIs' vertices and track parameters are very different from background ROIs' because of LLP's distant decay in the tracker volume.
We used ML for the multivariate analysis tool.
We used only the leading and the subleading scores of an event due to two LLP presence in a signal event.
Muon's IP significance value was also used for event selection.
We used the ABCD method to predict the background rate in the signal region.
We validated the method without unblinding by looking at the correlation factor at the $\Delta\phi$ inverted regions.
Systematic uncertainty originates from $\Lumi_{int}$ recording, lepton scale factors, background estimation, and statistical uncertainties.
With these, we used the HiggsAnalysis' CombineTools to set limits on the branching ratio of the Higgs to LLPs for $\tau$ lepton channel.
We observed no data over the SM expectation, set the limits on the branching ratio of the Higgs to LLPs, \textbf{B}(\PH$\to SS \to \tau^{+}\tau^{-}\tau^{+}\tau^{-}$), to a stringent value.
The exclusion limit is one of the most competitive results from CMS analyses, with the branching ratio set to a single digit for several combinations of MS and c$\uptau$.
